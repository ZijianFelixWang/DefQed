%% Copyright 2022 Zijian Wang. All rights reserved.
%% Email: felix_wzj@yahoo.com
%% The newest version can be obtained from GitHub.

\documentclass[preprint]{elsarticle}

\hfuzz=2pt
%% \special{dvipdfmx:config z 0} %% speed boost

%% The commands below, learnt from TeX stack exchange can cancel the hbox overfull warning...
\usepackage{etoolbox}
\makeatletter
\patchcmd\ps@pprintTitle
{\fi\fi\fi\fi}
{\fi\fi\fi\fi
	\afterassignment\fix@elsarticle\let\@tempa}
{}{\FailedToPatch}
\def\fix@elsarticle{\iffalse{\fi}\romannumeral-`0}
\makeatother

%% \usepackage[letterpaper, total={5.3in, 9in}]{geometry}	%% set paper size & margin
\usepackage{amsmath}	%% math part 1
\usepackage{flexisym, breqn}	%% math part 2
\usepackage{amssymb, graphics, setspace}	%% math part 3
\usepackage{tikz, tikz-cd}	%% for commutative diagrams
\usepackage{flowchart}	%% for flowchart
\usepackage{float, booktabs}	%% for table & alignment
\usepackage{algorithmicx, algorithm, algpseudocode}	%% for pseudo code composition
\usepackage{blindtext}	%% overfull hbox solution
\usepackage{setspace}	%% double spacing

\usetikzlibrary{positioning, cd, arrows}	%% for comutative diagram & flowchart drawing

\doublespacing

\begin{document}
	\begin{figure}[ht]
		\centering
		\begin{tikzpicture}[>=latex', font={\sf \small}]
			\def\smbwd{1cm}
			
			\node (start) at (0,0) [draw, terminal,
			minimum width = \smbwd,
			minimum height = 0.5cm] {Start};
			
			\node (parse) at (0,-1.5) [draw, process,
			minimum width = \smbwd,
			minimum height = 0.5cm] {Get task from user};
			
			\node (l-pool) at (-2,-3) [draw, storage,
			minimum width = \smbwd,
			minimum height = 0.5cm] {$\mathcal{M}_t$};
			
			\node (r-pool) at (2,-3) [draw, storage,
			minimum width = \smbwd,
			minimum height = 0.5cm] {$\mathcal{R}$};
			
			\node (rec-functor) at (-2,-4.5) [draw, predproc,
			minimum width = \smbwd,
			minimum height = 0.5cm] {$\mathcal{F}_t$};
			
			\node (random) at (-4,-4.5) [draw, process,
			minimum width = \smbwd,
			minimum height = 0.5cm] {$\mathit{r}$};
			
			\node (k-base) at (2,-4.5) [draw, storage,
			minimum width = \smbwd,
			minimum height = 0.5cm] {KBase $\mathcal{Q}$};
			
			\node (success-val) at (0,-6) [draw, process,
			minimum width = \smbwd,
			minimum height = 0.5cm] {$s = {\exists n\in [0,\omega ), \mathcal{R}\subseteq \text{Ob}\left(\mathcal{M}_n\right)}$};
			
			\node (success-check) at (0,-7.5) [draw, decision,
			minimum width = \smbwd,
			minimum height = 0.5cm] {$s$};
			
			\node (reinforcement) at (2,-9) [draw, predproc,
			minimum width = \smbwd,
			minimum height = 0.5cm] {Reinforcement};
			
			\node (write-rh) at (-2,-9) [draw, process,
			minimum width = \smbwd,
			minimum height = 0.5cm] {Generate steps};
			
			\node (steps) at (-4,-9) [draw, storage,
			minimum width = \smbwd,
			minimum height = 0.5cm] {Steps};
			
			\node (forgetful-para) at (2,-7.5) [draw, process,
			minimum width = \smbwd,
			minimum height = 0.5cm] {$E(\mathit{t})$};
			
			\node (end) at (0,-10.5) [draw, terminal,
			minimum width = \smbwd,
			minimum height = 0.5cm] {End};
			
			\coordinate (point1) at (-4.8,-7.5);
			\coordinate (point2) at (3.2,-4.5);
			\coordinate (point3) at (3.2,-8);
			%% \coordinate (point2) at (-7,-3);
			
			\draw[->] (start) -- (parse);
			\draw[->] (parse) |- (l-pool);
			\draw[->] (parse) |- (r-pool);
			\draw[->] (random) -- (rec-functor);
			\draw[->] (l-pool) -- (rec-functor);
			\draw[->] (k-base) -- (rec-functor);
			\draw[->] (l-pool) -| (success-val);
			\draw[->] (r-pool) -| (success-val);
			\draw[->] (success-val) -- (success-check);
			%% \draw[->] (success-check) -| node[above]{No} (l-pool);
			\draw[->] (success-check) -- node[above]{No} (point1);
			\draw[->] (point1) |- (l-pool);
			\draw[->] (success-check) |- node[below]{Yes} (write-rh);
			\draw[->] (success-check) |- node[below]{Yes} (reinforcement);
			\draw[->] (write-rh) -- (steps);
			\draw[->] (write-rh) |- (end);
			\draw[->] (forgetful-para) -- (reinforcement);
			%% \draw[->] (reinforcement) -- (k-base);
			\draw[->] (point2) -- (k-base);
			\draw[->] (point3) -- (point2);
			\draw[->] (reinforcement) |- (point3);
			\draw[->] (reinforcement) |- (end);
		\end{tikzpicture}
		\caption{The flowchart of the entire algorithm.}
		\label{figure-3}
	\end{figure}
\end{document}